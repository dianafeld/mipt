\documentclass[12pt]{article}

%\usepackage{extsizes}
%\usepackage{setspace}
\usepackage[T2A]{fontenc}
\usepackage[utf8]{inputenc}
\usepackage[russian]{babel}
\usepackage{indentfirst}

\usepackage{amsmath}
\usepackage{amssymb}
\usepackage{amsthm}
\usepackage{icomma}
\usepackage{multicol}
\usepackage{multirow}

\usepackage{graphicx}
\usepackage{epstopdf}


\theoremstyle{plain}
 \newtheorem{theorem}{Теорема}
 
 \theoremstyle{remark}
  \newtheorem*{remark}{Замечание}

\theoremstyle{definition} 
\newtheorem{definition}{Определение}

\newtheorem{lemma}[theorem]{Лемма}


\begin{document}



\begin{titlepage}

\begin{center}

\textsc{\Large Московский физико-технический институт}\\[0.5cm]
\textsc{\large Кафедра дискретной математики}\\[1cm]

\vfill


{ \LARGE  $\bigoplus \notin \mathbf{AC}^0$}
\end{center}

\vfill 

\begin{flushright} 
\begin{minipage}{0.5\textwidth}
\begin{flushright}
\normalsize
Работа студентки 3 курса, гр.~294 Ичаловой Дианы\\
\end{flushright}
\end{minipage}
\end{flushright}


\vfill

\begin{center}
{\large Москва\\
2014 г.}
\end{center}

\end{titlepage}


\section{Введение}

В 1970-х годах важным вопросом схемной сложности был вопрос получения нижних оценок на ресурсы, требуемые для вычисления различных функций. Считалось, что это поможет решить знаменитую проблему, доказав, что $P \ne NP$. В частности, рассматривался вопрос, какие функции можно выразить схемами из $\mathbf{AC}^0$. Ученые пытались найти функции, которые не лежат в классе схем с небольшой глубиной. Только в начале 80-х годов было доказано, что даже $\texttt{PARITY} \notin \mathbf{AC}^0$. Этот результат был получен в 1981 году [Furst, Saxe, Sipser] и независимо [Ajtai] в 1983 году. Впервые экспоненциальная оценка на  $\texttt{PARITY}$ была получена [Yao] в 1985 году. В 1987 году Хостад [H\aa stad] сформулировал и доказал лемму о переключении, которая позволила получить более точные нижние оценки для $\texttt{PARITY}$ и для других функций, например, $\texttt{MAJORITY}$.

Схемы с небольшой глубиной имеют также важное прикладное значение в теории параллельных вычислений. Они имеют отношение к PRAM \emph{(parallel random-access machine)}. Поэтому, получая нижние оценки на глубину схем, можно получить нижнюю оценку времени вычисления этой функции на PRAM.

% написать почему задача важна, кто и как ее доказал, про лемму Хостада PRAM


\section{Определения}
\begin{definition}

Для всех $n \in \mathbb{N}$, \emph{булевой схемой с $n$ входами и одним выходом} будем называть ориентированный ациклический граф с $n$ истоками (вершинами без входящих ребер), помеченными переменными из множества $\{x_1, . . . , x_n\}$, и одним стоком (вершиной без исходящих ребер).  Остальные вершины, помеченные символами $\vee, \wedge, \lnot$, будем называть \emph{функциональными элементами}. Вершины, помеченные  $\vee$ и $\wedge$, могут иметь произвольное число входящих ребер, а вершины, помеченные $\lnot$, имеют ровно одно входящее ребро. Каждый элемент вычисляет булеву функцию очевидным образом. \\
\emph{Размер схемы} "--- это число элементов в ней. \emph{Глубина элемента} "--- это максимальное число элементов на пути от данного элемента  до входа. \emph{Глубина схемы} "--- глубина выхода. \\



%\emph{Булевой схемой над базисом $\mathcal{B}$} называется ациклический ориентированный граф, у которого все вершины входящей степени 0 помечены  и называются \emph{входами}.
%Вершины входящей степени 2 помечены функциями из $\mathcal{B}$ и называются \emph{элементами}. Также есть $m$ специально выделенных \emph{выходных элементов}.
%Таким образом, схема вычисляет функцию из $\{0, 1\}^n \to \{0, 1\}^m$. 

\end{definition}

Везде далее, для простоты изложения, будем считать, что входными элементами являются не только сами переменные, но и их отрицания. Также, применяя закон де Моргана, можно построить эквивалентную схему, используя только операции $\vee$ и $\lnot$:
\[ A_1 \wedge A_2 \wedge \ldots \wedge A_n  = \overline{\overline{A_1} \vee \overline{A_2} \vee \ldots \vee \overline{A_n}}\]
Тогда размером схемы будем называть число элементов $\vee$, а глубиной элемента "--- максимальное число элементов $\vee$ на пути от данного элемента до входа.


\begin{definition}
Язык $L$ \emph{разрешим семейством булевых схем $\{C_n\}$}, если для всех $x \in \{0, 1\}^n$, 
\[x \in L \Leftrightarrow C_n(x) = 1\]
\end{definition}

\begin{definition}
Язык $L$ лежит в сложностном классе $\mathbf{AC}^0$, если он разрешим семейством булевых схем $\{C_n\}$, где $C_n$ имеет полиномиальный размер и константную глубину.
\end{definition}

\begin{definition}
$\bigoplus = \{x \in \{0, 1\}^* : \text{в } x \text{ нечетное число единиц} \}  $
\end{definition}

\begin{definition}
\emph{Ограничением} набора булевых переменных $\{x_i : i \in I \}$ называется отображение $\rho : I \to \{0, 1, *\}$. Результат применения ограничения к булевой функции $f$ есть булева функция $f|_\rho$, определяемая как результат подстановки $\rho(i)$ вместо $x_i$ для всех $i$ таких, что $\rho(i) \ne *$. Переменная $x_i$ называется \emph{неопределенной}, если $\rho(i) = *$.

Множество всех ограничений $n$  переменных с ровно $\ell$ неопределенными переменными будем обозначать $\mathcal{R}_n^\ell$.

На множестве всех ограничений $n$ переменных естественным образом определена операция произведения непересекающихся ограничений.

Будем считать, что после применения ограничения к ДНФ-формуле, формула упрощается путем удаления обнуленных конъюнктов.

\end{definition}

\begin{definition}
\emph{Дерево принятия решений $T(F)$} для формулы $F$ в ДНФ определяется индуктивно следующим образом:
\begin{enumerate} 
\item Если $F$ "--- это константный 0 или 1 ($F$ не содержит термов или первый терм пустой, соответственно), тогда $T(F)$ состоит из одного листа, помеченного соответствующей константой.
\item Если $F = C_1 \vee F'$, где терм $C_1$ не пуст, то пусть $K$ "--- множество переменных в $C_1$. Дерево $T(F)$ начинается с полного бинарного дерева для $K$, которое запрашивает переменные $K$ в порядке увеличения их индексов. Каждый лист $v_\sigma$ соответствует ограничению $\sigma$, которое присваивает переменным из $K$ значения в соответствии с путем в дереве от корня до $v_\sigma$. Для каждого $\sigma$ заменим лист $v_\sigma$ на дерево принятия решений для  $T(F|_\sigma)$. \\
Отметим, что для единственного $\sigma$, удовлетворяющего терм $C_1$, лист $v_\sigma$ станет листом, помеченным 1, все остальные $v_\sigma$ будут заменены на поддеревья $T(F|_\sigma) = T(F'|_\sigma)$.
\end{enumerate}
Высоту полученного дерева будем обозначать $|T(F)|$.
\end{definition}

\section{Лемма Хостада о переключении}
Данная лемма и ее многочисленные вариации являются мощным иструментом для доказательства нижних оценок размеров схем для различных функций.

В оригинальном доказательстве \cite{hastad} Йохан Хостад использовал условные вероятности. Позднее, более простое доказательство было предложено Александром Разборовым в его работе \cite{Razborov}, оно использует понятия минтермов и макстермов КНФ и ДНФ. Доказательство, приведенное ниже, повторяет \cite{beame}, которое оперирует более наглядным деревом принятия решений.

Прежде чем формулировать теорему, докажем одну несложную лемму.

\begin{definition}
Определим $stars(r, s)$ как множество всех последовательностей $\beta = (\beta_1, \ldots, \beta_k)$ произвольной длины таких, что выполнены условия:
\begin{enumerate}
\item Для всех $j$: $\beta_j \in \{-, *\}^r \setminus \{-\}^r$
\item Суммарное число $*$ во всех $\beta_j$ равно $s$.
\end{enumerate}
\end{definition}

\begin{lemma}
\[|stars(r, s)|  < \left( \frac{r}{\ln2}\right)^s \]
\end{lemma}
\begin{proof}
Доказательство индукцией по s, что $|stars(r, s)| \le \gamma^s$ для $\gamma$, удовлетворяющего равенству $(1 + \frac1\gamma)^r = 2$. Прежде чем проводить индукцию, найдем отсюда оценку на $\gamma$:
\[ 2^{\frac1r} = 1 + \frac1\gamma < e^\frac1\gamma,\]
где последнее неравенство следует из формулы Тейлора.
Откуда
\[ e^{\frac{\ln2}{r}}  < e^{\frac1\gamma},\]
\[ \gamma < \frac{r}{\ln2}. \] 

\textbf{База}
$s = 0$ \\ $|stars(r, 0)| = 1 \le \gamma^0$ ($stars(r, 0)$ содержит пустую строку).

\textbf{Предположение индукции} \\
Предположим, что для всех $s < t$ $|stars(r, s)| \le \gamma^s$.

\textbf{Шаг индукции} \\
Докажем для $s = t$, что  $|stars(r, s)| \le \gamma^s$. Рассмотрим первый элемент последовательности $\beta$ "--- $\beta_1$. Пусть он содержит$k$ $*$.  Тогда последовательность $\beta$ без элемента $\beta_1$ лежит в множестве $stars(r, s - k)$. Так как $\beta_1$ можно выбрать $C_r^k$ способами, то
\begin{gather*}
 |stars(r, s)| = \sum_{k = 1}^{\min(r, s)} C_r^k|stars(r, s - k)| \le \\ \le \sum_{k = 1}^{r} C_r^k \gamma^{s - k} = \gamma^s \sum_{k = 1}^r C_r^k \left(\frac1\gamma\right)^k = \\ = \gamma^s \left[\left(1 + \frac1\gamma\right)^r - 1\right] = \gamma^s [2 - 1] = \gamma^s.
 \end{gather*}

Из этого и из определения $\gamma$ следует, что 
\[  |stars(r, s)| \le \gamma^s < \left( \frac{r}{\ln2}\right)^s \]
 
\end{proof}

\begin{theorem}[H\aa stad's switching lemma]
Пусть $F$ "--- $r\text{--ДНФ}$ формула от $n$ переменных, $\text{Bad}(F, s)$ "--- множество ограничений $\rho \in \mathcal{R}_n^\ell $, для которых  $|T(F|_\rho)| \ge s$. Тогда для любого $s \ge 0$, $\ell = pn$ и $p \le 1/7$
\[ \frac{|Bad(F, s)|}{|\mathcal{R}_n^\ell |} < (7pr)^s.\]

\end{theorem}

\begin{proof}
Лемма доказывается построением биекции: $Bad(F, s) \to \mathcal{R}_n^{\ell - s} \times stars(r, s) \times 2 ^ s$.

\textbf{Инъективность} 

Докажем инъективность, предъявив для каждого ограничения $\rho \in \mathcal{R}_n^{\ell}$ элемент из $\mathcal{R}_n^{\ell - s} \times stars(r, s) \times 2 ^ s$ однозначным образом.

Зафиксируем формулу $F = \bigvee_{i = 1}^H C_i$, где $C_i$ "--- некоторые дизъюнкты. Обозначим $\left\{ x_1, \ldots, x_n \right\}$ "--- множество переменных $F$. Пусть $\rho \in Bad(F, s)$ и пусть $\pi$ "--- некоторый путь в дереве  $T(F|_\rho)$, длина которого больше или равна $s$. Если $\pi$ присваивает значения больше, чем $s$ переменным, то обрежем $\pi$ до первых $s$ переменных. %  $\pi$ "--- лексикографически наименьший путь в дереве  $T(F|_\rho)$

Рассмотрим первый терм $C_{\nu_1}$, который  не обнуляется под действием $\rho$. Такой обязательно найдется, так как дерево принятия решений $T(F|_\rho)$ не вырождается в лист. Пусть $K$ "--- множество переменных, содержащихся в $C_{\nu_1}|_\rho$. Определим $\sigma_1$ "--- ограничение переменных $K$, которое обращает терм $C_{\nu_1}|_\rho$ в 1 ($\sigma_1$ определена единственным образом). Определим
\[ \pi_1(i) = \begin{cases} \pi(i), \text{ если } x_i \in K, \\
  *, \text{ иначе}  \end{cases} \]

Тогда есть 2 случая:
\begin{enumerate}
\item $\pi_1 \ne \pi$. Тогда из построения дерева принятия решений и ограничения $\pi$ следует, что  $\pi_1$ определяет все переменные в $K$. Очевидно, $\pi_1 \ne \sigma_1$, так как иначе $C_{\nu_1}|_{\rho\pi_1} = 1$ и $T(F|_{\rho\pi_1})$ вырождается в лист, чего не может быть так как $\pi_1 \ne \pi$. Следовательно, $C_{\nu_1}|_{\rho\pi_1} = 0$.
\item $\pi_1 = \pi$. Обрежем $\sigma_1$ так, чтобы в нем содержались лишь переменные, которые содержатся в $\pi_1$. Тогда $C_{\nu_1}|_{\rho\sigma_1}$ не обращается в ноль.
\end{enumerate}

Определим $\beta_1 \in \{-, *\}^r \setminus \{-\}^r$: $j$-ая компонента $\beta_1$ равна $*$ тогда и только тогда, когда $\sigma_1$ определяет $j$-ую переменную в  $C_{\nu_1}$. Таким образом, зная $C_{\nu_1}$ и $\beta_1$, можно восстановить $\sigma_1$ следующим образом: рассмотреть только те переменные $x_{{\nu_1}_j}$, для которых $\beta_1^j = *$ и если $x_{{\nu_1}_j}$ входит без отрицания в  конъюнкт $C_{\nu_1}$, то положить $\sigma({\nu_1}_j) = 1$ и 0 иначе.

Если $\pi_1 \ne \pi$, то рассмотрим $\pi \setminus \pi_1$, которое является корректным ограничением в дереве $T(F|_{\rho\pi_1})$ и повторим рассуждения выше для $\widetilde{\pi} = \pi \setminus \pi_1$, $\widetilde{\rho} = \rho\pi_1$ и рассматривая первый терм  $C_{\nu_2}$ не обнуляющийся под действием $\widetilde{\rho}$. Таким образом мы получаем, что $\pi = \pi_1\pi_2\ldots \pi_k$, $\sigma = \sigma_1\sigma_2\ldots \sigma_k$, $\beta = \left( \beta_1, \beta_2, \ldots, \beta_k \right)$. 

Определим вектор $\delta \in \{0, 1\}^s$, который показывает, равны ли соответствующие значения переменных, определенные ограничениями $\pi$ и $\sigma$.

Итак, каждому $\rho$ мы поставили в соответствие тройку $<\rho\sigma, \beta, \delta>$, где $\rho\sigma \in \mathcal{R}_n^{\ell - s}$, $\beta \in stars(r, s)$, $\delta \in \{0, 1\}^s$.

\textbf{Cюръективность} 

Покажем, как по тройке $<\rho\sigma = \rho\sigma_1\sigma_2\ldots \sigma_k, \beta = \left( \beta_1, \beta_2, \ldots, \beta_k \right), \delta>$ восстановить $\rho$.

Будем восстанавливать $\rho$ итеративно.
Пусть на $i$-ом шаге уже восстановлены $\pi_1, \ldots, \pi_{i -1}$, $\sigma_1, \ldots, \sigma_{i - 1}$ и построено $\rho\pi_1\ldots\pi_{i -1}\sigma_i\ldots\sigma_k$. 

Заметим, что для всех $i < k$ $C_{\nu_i}|_{\rho\pi_1\ldots\pi_{i-1}\sigma_i\sigma_{i+1}\ldots\sigma_k} = 1$ и $C_j|_{\rho\pi_1\ldots\pi_{i-1}\sigma_i\sigma_{i+1}\ldots\sigma_k} = 0$ для всех $j < \nu_i$.  

Если же $i = k$, то $C_{\nu_i}|_{\rho\pi_1\ldots\pi_{i-1}\sigma_i\sigma_{i+1}\ldots\sigma_k} \ne 0$ и $C_j|_{\rho\pi_1\ldots\pi_{i-1}\sigma_i\sigma_{i+1}\ldots\sigma_k} = 0$. \\ Тогда можно восстановить $C_{\nu_i}$ как индекс первого терма, который не обнуляется под действием $\rho\pi_1\ldots\pi_{i -1}\sigma_i\ldots\sigma_k$.
Как было описано выше по $C_{\nu_i}$ и $\beta_i$ мы можем восстановить $\sigma_i$. По $\sigma_i$ и $\delta$ восстанавливаем $\pi_i$. Зная переменные, которые присваивает $\sigma_i$, мы можем из $\rho\pi_1\ldots\pi_{i -1}\sigma_i\ldots\sigma_k$ построить $\rho\pi_1\ldots\pi_{i -1}\pi_i\ldots\sigma_k$. В конце концов, зная все $\pi_i$ и $\rho\pi_1\ldots\pi_{i -1}\pi_i\ldots\pi_k$ можно восстановить $\rho$.

\textbf{Получение верхней оценки}

Очевидно, что $\mathcal{R}_n^{\ell} = C_n^\ell  2 ^{n - \ell}$. (Выбираем $\ell$ неопределенных переменных, остальные переменные полагаем 0 или 1.)
Тогда
\begin{gather*} \frac{|\mathcal{R}_n^{\ell - s}|}{|\mathcal{R}_n^{\ell}|} = \frac{n!}{(\ell - s)!(n - \ell + s)!} \cdot \frac{\ell!(n - \ell)!}{n!} \cdot \frac{2^{n - \ell + s}}{2^{n - \ell}} = \\ = \frac{\ell  (\ell - 1) \ldots (\ell - s + 1) }{(n - \ell + s)(n - \ell + s - 1)(n - \ell + 1) } \cdot 2^s \le \frac{\ell^s}{(n - \ell)^s} \cdot 2^s.
\end{gather*}

Применяя лемму 1, получаем 
\begin{gather*}
\frac{|Bad(F, s)|}{|\mathcal{R}_n^{\ell}|} = \frac{|\mathcal{R}_n^{\ell - s}|}{|\mathcal{R}_n^{\ell}|} \cdot |stars(r, s)| \cdot 2^s < \\ < \frac{(2\ell)^s}{(n - \ell)^s} \cdot  \left( \frac{r}{\ln2}\right)^s \cdot 2^s = \left( \frac{4\ell r}{(n - \ell)\ln2} \right)^s.
\end{gather*}

Учитывая, что $\ell = pn$ и $p < 1/7$, окончательно получаем, что 
\[ \frac{|Bad(F, s)|}{|\mathcal{R}_n^{\ell}|} < (7pr)^s. \]

\end{proof}

\section{$\bigoplus \notin \mathbf{AC}^0$ }
Доказательство использует вероятностный метод: рассматриваются случайные ограничения и доказывается, что некоторое <<хорошее>> ограничение существует.

\begin{lemma}
Пусть $C$ "--- булева схема и $|C|$ "--- её размер, а $d$ "--- глубина. Определим $n_i = \dfrac{n}{14} \dfrac{1}{(14 \log_2|C|)^{i - 1}}$ для всех $1 \le i \le d$.\\
Тогда если $n_i \ge \log_2|C|$, то существует ограничение $\rho_i \in \mathcal{R}_n^{n_i}$ такое, что для любого функционального элемента $g$ на глубине не больше $i$ в $C$, $g|_{\rho_i}$ представимо в виде дерева принятия решений высоты не больше $\log_2|C|$.
\end{lemma}
\begin{proof}
Напомним, что входными элементами булевой схемы являются переменные и их отрицания, в схеме используются только операции $\vee$ и $\lnot$. В подсчете глубины схемы элементы $\lnot$ не учитываются.

Достаточно доказать теорему для элементов $g = \vee$, так как $\lnot g$ имеет такое же дерево принятия решений, как $g$, но с инвертированными значениями на листьях.

Докажем теорему индукцией по глубине $d$ схемы $C$.

\textbf{База} $d = 1$, $n_1 = \frac{n}{14}$ 

Входами элемента $\vee$ на глубине 1 являются переменные и их отрицания, следовательно, каждый такой элемент $g$ задает 1-ДНФ формулу. 

Положим $p = 1/14$, $n_1 = np$. По лемме о переключении число ограничений $\rho \in \mathcal{R}_n^{n_1}$ таких, что  $|T(g|_\rho)| \ge \log_2|C|$ строго меньше $(7 p \cdot 1)^{\log_2|C|} = (1/2)^{\log_2|C|} = 1/|C|$. Так как число элементов на глубине 1 не может превосходить общего числа элементов |C|, то найдется такое ограничение $\rho_1 \in \mathcal{R}_n^{n_1}$, что для всех элементов $g$ на глубине 1 $|T(g|_{\rho_1})| \le \log_2|C|$.

\textbf{Предположение индукции}

Предположим, что для всех $d < i$ существует ограничение $\rho_d \in \mathcal{R}_n^{n_d}$ такое, что для всех элементов $g$ на глубине не больше чем $d$, $|T(g|_{\rho_d})| \le \log_2|C|$. 

\textbf{Шаг индукции}

Докажем утверждение для $d = i$. \\
По предположению индукции существует $\rho_{i - 1} \in \mathcal{R}_n^{n_{i - 1}}$ такое, что для всех элементов $g$ на глубине не больше чем $i - 1$, $|T(g|_{\rho_{i - 1}})| \le \log_2|C|$. Тогда $g|_{\rho_{i - 1}}$ можно представить в виде ($\log_2|C|$)-ДНФ формулы. Рассмотрим элемент $g = \vee$ на глубине $i$ и применим ограничение $\rho_i$. Так как все входы этого элемента могут быть представлены в виде ($\log_2|C|$)-ДНФ формул, то и $g$ представима в виде ($\log_2|C|$)-ДНФ формулы. 

Положим $p = n_{i}/n_{i - 1} = 1/(\log_2|C|)$. По лемме о переключении число ограничений  $\pi \in \mathcal{R}_{n_{i - 1}}^{n_{i}}$ таких, что  $|T((g|_{\rho_{i - 1}})|_\pi)| \ge \log_2|C|$ строго меньше $(7 p \log_2|C|)^{\log_2|C|} = (1/2)^{\log_2|C|} = 1/|C|$. Так как на уровне $i$ не больше, чем $|C|$ элементов, то найдется такое ограничение $\pi$, что $|T(g|_{\rho_{i - 1}\pi})| \le \log_2|C|$. Полагая $\rho_i = \rho_{i - 1}\pi \in \mathcal{R}_n^{n_i}$, получаем требуемое ограничение.


\end{proof}

\begin{theorem}
\[ \bigoplus \notin \mathbf{AC}^0 \]
\end{theorem}

\begin{proof}
Докажем, что любая схема $C$ константной глубины, вычисляющая $\bigoplus$ имеет размер $|C| \ge 2^{\frac{1}{14}n^{\frac{1}{d}}}$. А так как $\mathbf{AC}^0$ содержит схемы с полиномиальным размером, то отсюда будет следовать условие теоремы.

Рассмотрим некоторую схему $C$ глубины $d$, не зависящей от длины входа $n$. Заметим, что для любого ограничения $\rho \in \mathcal{R}_n^{\ell}$ глубина каждой ветви дерева принятия решений для $\bigoplus$ равна $\ell$.  Следовательно, $|T(g|_{\rho})| = \ell$.


Применим ограничение $\rho_d$ из предыдущей леммы к схеме $C$. Тогда для любого элемента $g$ (и в частности для выходного) $|T(g|_{\rho_d})| \le \log_2|C|$. Но $|T(g|_{\rho_d})|  = n_d = \dfrac{n}{14^d\log_2^{d - 1}|C|}$. \\ Отсюда получаем неравенство:
\begin{align*}
\dfrac{n}{14^d\log_2^{d - 1}|C|} &\le \log_2|C|, \\
\log_2^{d}|C| &\ge \frac{n}{14^d}, \\
|C| &\ge 2^{\frac{1}{14}n^{\frac{1}{d}}}.
\end{align*}
\end{proof}
\nocite{*}

\newpage

\bibliographystyle{plain}
\bibliography{ParityAC}

\end{document}